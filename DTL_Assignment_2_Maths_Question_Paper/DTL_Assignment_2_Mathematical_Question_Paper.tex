\documentclass[a4paper,12pt,oneside,article]{memoir}

\setlrmarginsandblock{1.5cm}{*}{1}
\setulmarginsandblock{1.5cm}{*}{1}
\setheadfoot{2cm}{\footskip}          
\checkandfixthelayout[nearest]
\setlength{\parindent}{0pt}
\documentclass{article}
\usepackage{amsfonts}
\usepackage{amsmath}

\begin{document}

\begin{center}
  \bfseries\large
 College of Engineering, Pune.
 
  B.Tech II Year

  November 2022, Odd Semester
  \begin{center}
      TEST - 1
  \end{center}

  Linear Algebra and Uni-variate Calculus \qquad

  \bigskip

  \normalfont\normalsize
  Duration-1.5 hours     Marks-30 
\end{center}

\hrulefill

\emph{\textbf{All Questions Compulsory}}

\emp{\textbf{Use of Programmable Calculator Forbidden }}

\pagenumbering{gobble}
\begin{flushleft}
Q.1) Solve the following:\\[10pt] 
(a)$5y(xy-8)dy+(y^3+8y)dy=0$
\newline
(b)$(2\tan{}{y} + 2x^2)dx -(x\sin{y})dy==0$
\end{flushleft} 
\begin{flushright} [3x2=6]
\end{flushright}



\begin{flushleft}
 Q.2) Find a homogeneous linear second order ordinary differential equation whose solution is the set of all straight lines in the $xy$-plane.\\ 
\end{flushleft}

\begin{flushright} [2]
\end{flushright}

\begin{flushleft}
 Q.3)Which of the following forms sub-spaces. Provide or prove counter examples
\begin{equation*}
	\{(x,y)\in \R \mid x=y+2\} 
\end{equation*}

\begin{equation*}
	\{(x,y,z) \in \R \mid  x=y  , 2z =z\} 
\end{equation*}

\end{flushleft}

\begin{flushright} [3x2=6]
\end{flushright}

\begin{flushleft}
Q.4) If $x^2$ and $1$ are solutions of $yy''-xy'=0$ then so is any linear combination of these. State true or false and justify.
\end{flushleft}

\begin{flushright} [3]
\end{flushright}

\begin{flushleft}
Q.5) Find a linear ordinary differential equation for which the function $e^{-x}\cos{2x}$ and $e^{-x}\sin{2x}$ are linearly independent solutions.
\end{flushleft}
\begin{flushright} [3]
\end{flushright}


\begin{flushleft}
Q.6) Find the Inverse : 

\vspace{.25cm}

\centering
A = 
\begin{bmatrix} 
-1 & 0 & 8 \\
0 & -5 & 5\\
-2 & 0 & -1 \\
\end{bmatrix}
\quad
\end{flushleft}

\begin{flushright}
    [5]
\end{flushright}

\begin{flushleft}
    Q.7) Determine the values of a and b:
    \newline
    i. No-solution.
    ii. Infinite number of solutions 
    iii. Unique solution 
\begin{center}
\begin{equation}
	2x-y+4z=6
\end{equation}
\begin{equation}
        x+y-2z=0
\end{equation}
\begin{equation}
        3x+2y+1z=3      
\end{equation}

\end{center}
    
\end{flushleft}
\begin{flushright}
    [5]
\end{flushright}

\begin{flushleft}
Q.6)
	Find Inverse:
	
\begin{flushleft}

\begin{equation*}

A = \begin{bmatrix}

3 & 0 & 0  &  -1 \\

-1 & 0 & 3  & 0 \\

a1 & b & c4 & 3 \\

-1 & 0 & 3 & 2 \\

\end{bmatrix}

\end{equation*}

\begin{equation*}

B = \begin{bmatrix}

3 & 0 & 0  &  -1 \\

-1 & 0 & 3  & 0 \\

a1 & b & c4 & 3 \\

-1 & 0 & 3 & 2 \\

\end{bmatrix}

\end{equation*}

\end{flushleft} 

\begin{flushright}

[ 2 x 4 =8 ]

\end{flushright}



\newpage

\section{Assignment 2: Mathematical Expressions in Latex:}

\hrule

\vspace{2}

Equations \emph{Expressions} in Latex.

Equation 1:
\begin{equation}
	2x^2+4=6.\label{eq:ligning}
\end{equation}

Equation 2: 
\begin{flushright}
    $f(x) = a_{n} x^{n} + a_{n-1}x^{n-1}+\cdots +a_{1}x + a_{0}$.
\end{flushright}


Multiplication:

\begin{equation*}
	a\cdot b
\end{equation*}

\vspace{.005cm}

Fraction:


\begin{equation*}
	\frac{a}{b}
\end{equation*}

\vspace{.1cm}

Definite Integral:

\begin{equation*}
	\int_a^b f(x)\, dx. 
    f(x) og dx
\end{equation*}


\vspace{.1cm}

Summation:

\begin{equation*}
	\sum_{i=1}^n f(x_i) \Delta x
\end{equation*}


\vspace{.1cm}

Limits:

\begin{equation*}
	\lim_{\Delta \to 0} \frac{f(x_0+\Delta x)-f(x_0)}{\Delta x}
\end{equation*}

\vspace{.1cm}

Determinants:

\begin{equation*}
	\lvert -5 \rvert=5
\end{equation*}

\vspace{.1cm}

Root:

\begin{equation*}
	\sqrt{x+1}
\end{equation*}

\vspace{.1cm}

Vector:

\begin{equation*}
	\vec{a} =\begin{pmatrix} 1\\2\\3 \end{pmatrix} \text{, } \widehat{\vec{a}} \text{ og } \overrightarrow{AB}
\end{equation*}


\vspace{.1cm}

Sets:


\begin{equation*}
	\{x \in \R \mid 2\leq x<5\} 
\end{equation*}


\vspace{.1cm}

Sets:

\begin{equation*}
	f(x) = \begin{cases}
	x^2 & \text{ } x>2,\\
	x-1 & \text{ } x\leq 2
	\end{cases}
\end{equation*}


\vspace{.1cm}

Bracket Fraction:

\begin{equation*}
	\left(\frac{a}{b} \right)
\end{equation*}

\vspace{.1cm}

Matrices: 
 
\[A=
\begin{bmatrix}
6&9&3\\
5&2&1\\
4&8&7\\

\end{bmatrix}
\]

\[B=
\begin{bmatrix}
1&0&\cdots&0\\
1&0&\cdots&0\\
\vdots & \vdots &\ddots &\vdots\\
1&0&0&0 \\
\end{bmatrix}
\]

Mathematical Symbols:

\begin{center}
\begin{equation*}
	\pm \quad \infty \quad \leq \quad \geq \quad \circ \quad \in \quad \notin \quad \neq \quad \bullet \quad \Leftrightarrow \quad \Updownarrow \quad \times \quad \angle
\end{equation*}
\end{center}

\vspace{.5cm}

Trigonometry, Logarithm and Exponential :

\begin{equation*}
	\sin(x) \quad \cos(x) \quad \tan(x) \quad \ln(x) \quad \log(x) \quad \exp(x)
\end{equation*}

\vspace{.5cm}

\section* {\uline{Formula Derivation}}
Identities:

\begin{align*}
	(a+b)^2 &= (a+b)(a+b)\\
	&= a^2+ab+ba+b^2\\
	&=a^2+2ab+b^2
\end{align*}


\vspace{.5cm}

\section *{ \uline{Problem Statement :}}
Find a and b \textcolor{red}{Assume the Values}, \textbf{}  \textit{} \LaTeX. 
\begin{equation*}
	\textcolor{blue}{a^2}+\textrm{b}^2=\bm{c^2} \text{ kg } 
 \vspace{.1cm}
 \end{equation*}
 
 \begin{equation*}
  \uline{a+b}=\uuline{\SI{4,56e4}{kg.m^2.s^{-3}}{}}.
 \end{equation*}



\end{document}